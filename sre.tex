\chapter{Einleitung}

Die erste Quelle \cite{survey}.

\section{Überblick  zu Rubin und Chechik}

\begin{itemize}
	\item Feature $=$ Software Artefakt, das eine spezifische Funktionalität implementiert
	\item SPLE: Software Product Line Engineering
	\item Tracebility der Features ist eine der Kernaufgaben von SPLE
	\item Essenziell für Wartbarkeit von Programmen
	\item Feature Location hat als Ziel die Identifikation der Beziehung zwischen Features und Implementierung
	\item Feature nach Rajlich und Chen besteht aus Name, Insension/Bedeutung und Erweiterung(Extension)
	\item Grundlagen:
	\begin{itemize}
		\item Formal Concept Analysis (FCA)
		\item Latent Semantic Indexing (LSI)
		\item Term Frequency - Inverse Document Frequency Matrix
		\item Hyper-link Induced Topic Search (HITS)
	\end{itemize}
	\item Technologien:
	\begin{itemize}
		\item Statische Feature Location Technologien
		\begin{itemize}
			\item Plain Output
			\item Guided Output
		\end{itemize}
		\item Dynamische Feature Location Technologien
		\begin{itemize}
			\item Plain Output
			\item Guided Output	
		\end{itemize}
	\end{itemize}
\end{itemize}

\chapter{Analyseverfahren}
\section{Statische Analyse}
\section{Dynamische Analyse}
\section{Textuelle Analyse}

\chapter{Tools}
\chapter{Beispiel}
\chapter{Fazit}