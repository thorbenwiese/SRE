
%%%%%%%%%%%%%%%%%%%%%%% file typeinst.tex %%%%%%%%%%%%%%%%%%%%%%%%%
%
% This is the LaTeX source for the instructions to authors using
% the LaTeX document class 'llncs.cls' for contributions to
% the Lecture Notes in Computer Sciences series.
% http://www.springer.com/lncs       Springer Heidelberg 2006/05/04
%
% It may be used as a template for your own input - copy it
% to a new file with a new name and use it as the basis
% for your article.
%
% NB: the document class 'llncs' has its own and detailed documentation, see
% ftp://ftp.springer.de/data/pubftp/pub/tex/latex/llncs/latex2e/llncsdoc.pdf
%
%%%%%%%%%%%%%%%%%%%%%%%%%%%%%%%%%%%%%%%%%%%%%%%%%%%%%%%%%%


\documentclass[runningheads,a4paper]{llncs}

\usepackage{amssymb}
\setcounter{tocdepth}{3}
\usepackage{graphicx}
\usepackage[utf8]{inputenc}
\usepackage[german]{babel}
\usepackage[numbers]{natbib}

\usepackage{url}
\newcommand{\keywords}[1]{\par\addvspace\baselineskip
\noindent\keywordname\enspace\ignorespaces#1}

\begin{document}

\title{Einstiegspunkte für Design und Codierung \\bei einer Wartungsaufgabe oder Fehlermeldung}

%\titlerunning{Einstiegspunkte für Design und Codierung bei einer Wartungsaufgabe oder Fehlermeldung}

\author{Felix Fröhlich \and Thorben Wiese}

%\authorrunning{Felix Fröhlich \and Thorben Wiese}

\institute{Universität Hamburg \\
Fakultät für Mathematik,\\
Informatik und Naturwissenschaften \\
Department Informatik}

\maketitle

\begin{abstract}
Bei der Entwicklung und Wartung von Software spielt die Identifikation des Nutzen von Programmabschnitten eine große Rolle, um entsprechende Funktionen einer Software zu ändern oder zu reparieren. Diese \textit{Feature Locations} stellen einen Einstigespunkt in den Quelltext für Design- oder Code-Änderungen dar und können mithilfe verschiedener Technologien ermittelt werden. In dieser Seminararbeit stellen wir unterschiedliche Technologien und deren Verfahren vor und geben einen Überblick über geeignete Nutzungsfelder.
\end{abstract}

\section{Einleitung}

%TODO Am Ende schreiben?
Die erste Quelle \cite{survey}.\\
...\\
Ziel dieser Seminararbeit ist es, die verschiedenen Analyseverfahren und Technologien zur Erkennung von Features im Code zu beschreiben und zu vergleichen.

\section{Begriffe}

Für die Vorstellung der Analyseverfahren sollen zunächst einige Begriffe definiert und erklärt werden.

\subsection*{Feature}
Ein Feature ist ein Software Artefakt, das eine spezifische Funktionalität implementiert \cite{feature}. Diese Funktionalität wird in natürlicher Sprache beschrieben und wird von einem Programmabschnitt wiedergespiegelt. Ein Feature besteht üblicherweise aus einen Namen, einer Bedeutung (Intension) und einer Erweiterung (Extension) \cite{rajlich-chen}.

\subsection*{Feature Location}
Der Prozess der Feature Location beschreibt die Identifikation der Beziehung zwischen Features und deren Implementierung. Dabei liegt die Beschreibung des Features in natürlicher Sprache vor, die dann einem entsprechenden Codeabschnitt zugeordnet werden soll (Mapping) \cite{survey}.

\section{Analyseverfahren}

In diesem Abschnitt sollen verschiedene Analyseverfahren vorgestellt werden, die das Finden von Feature Locations ermöglichen.

%\subsection*{Formal Concept Analysis (FCA)}

%Thorben
\subsection*{Program Dependence Analysis (PDA)}

%Felix
\subsection*{Trace Analysis}

%Thorben
\subsection*{Latent Semantic Indexing (LSI)}


%\subsection*{Term Frequency - Inverse Document Frequency Matrix}

%\subsection*{Hyper-link Induced Topic Search (HITS)}

% technologies: PDA, LSI, Trace Analysis

%Thorben
\subsection{Statische Feature Location Technologien}

In diesem Abschnitt sollen statische Technologien zum Finden von Features im Quelltext vorgestellt werden.

\subsection*{Statische Analyse}

Die Analyse von Quelltext zu einem Zeitpunkt, bei dem das Programm kompiliert wird und noch nicht ausgeführt wurde, wird statische Analyse genannt. Hierbei werden mithilfe von zum Beispeil Datenfluss-Analyse und Kontrollgraphen alle Abhängigkeiten und Funktionsaufrufe innerhalb des Codes analysiert und es können unter anderem Fehler wie zum Beispiel Race-Conditions oder Buffer-Overflows identifiziert werden. Dieser Prozess wird häufig von automatisierten Tools durchgeführt \cite{static}.

\subsection*{Technologie Beispiele}

Robillard et al. [35]\\
Shao et al. [40]

%Felix
\subsection{Dynamische Feature Location Technologien}

\subsection*{Dynamische Analyse}
Allgemein dynamische Analyse

\subsection*{Technologie Beispiele}
Wong et al. [49]\\
Liu et al. [25] (SITIR)

\subsection{Textuelle Feature Location Technologien}
Allgemein textuelle Analyse, Tools beschreiben, Beispiele

\section{Vergleich}

\section{Fazit}
Für welchen Zweck welches Analyseverfahren und welche Technologie
Ergebnis wahrscheinlich: Alles gar nicht so schlecht, je nach Bedarf muss eine Technologie ausgewählt werden oder eventuell mit einer anderen kombiniert werden.



%%%%%%%%%%%%%%%%%%%%%%% %%%%%%%%%%%%%%%%%%%%%%%%%%%%%%%%%%%
%												LITERATURVERZEICHNIS 												       %
%%%%%%%%%%%%%%%%%%%%%%%%%%%%%%%%%%%%%%%%%%%%%%%%%%%%%%%%%%

\clearpage

\phantomsection % benötigt für korrekte pdf-darstellung
\addcontentsline{toc}{section}{Literaturverzeichnis}
\bibliographystyle{vancouver} % Din 1505 nach Lorenzen (Das konkrete Aussehen des Litverzeichnisses ist im header festgelegt)
\setcitestyle{square}
\bibliography{literatur}  % Pfad zur *.bib-Datei (Dateiendung wird weggelassen)

\end{document}
