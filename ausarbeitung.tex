
%%%%%%%%%%%%%%%%%%%%%%% file typeinst.tex %%%%%%%%%%%%%%%%%%%%%%%%%
%
% This is the LaTeX source for the instructions to authors using
% the LaTeX document class 'llncs.cls' for contributions to
% the Lecture Notes in Computer Sciences series.
% http://www.springer.com/lncs       Springer Heidelberg 2006/05/04
%
% It may be used as a template for your own input - copy it
% to a new file with a new name and use it as the basis
% for your article.
%
% NB: the document class 'llncs' has its own and detailed documentation, see
% ftp://ftp.springer.de/data/pubftp/pub/tex/latex/llncs/latex2e/llncsdoc.pdf
%
%%%%%%%%%%%%%%%%%%%%%%%%%%%%%%%%%%%%%%%%%%%%%%%%%%%%%%%%%%%%%%%%%%%


\documentclass[runningheads,a4paper]{llncs}

\usepackage{amssymb}
\setcounter{tocdepth}{3}
\usepackage{graphicx}
\usepackage[utf8]{inputenc}
\usepackage[german]{babel}
\usepackage{natbib}

\usepackage{url}
\newcommand{\keywords}[1]{\par\addvspace\baselineskip
\noindent\keywordname\enspace\ignorespaces#1}

\begin{document}

\title{Einstiegspunkte für Design und Codierung \\bei einer Wartungsaufgabe}

\titlerunning{Einstiegspunkte für Design und Codierung bei einer Wartungsaufgabe}

\author{Felix Fröhlich \and Thorben Wiese}

\authorrunning{Felix Fröhlich \and Thorben Wiese}

\institute{Universität Hamburg \\
Fakultät für Mathematik,\\
Informatik und Naturwissenschaften \\
Department Informatik}

\maketitle

\begin{abstract}
The abstract should summarize the contents of the paper and should
contain at least 70 and at most 150 words. It should be written using the
\emph{abstract} environment.
\end{abstract}

\section{Einleitung}

Die erste Quelle \cite{survey}.

\section{Überblick  zu Rubin und Chechik}

\begin{itemize}
	\item Feature $=$ Software Artefakt, das eine spezifische Funktionalität implementiert
	\item SPLE: Software Product Line Engineering
	\item Tracebility der Features ist eine der Kernaufgaben von SPLE
	\item Essenziell für Wartbarkeit von Programmen
	\item Feature Location hat als Ziel die Identifikation der Beziehung zwischen Features und Implementierung
	\item Feature nach Rajlich und Chen besteht aus Name, Insension/Bedeutung und Erweiterung(Extension)
	\item Grundlagen:
	\begin{itemize}
		\item Formal Concept Analysis (FCA)
		\item Latent Semantic Indexing (LSI)
		\item Term Frequency - Inverse Document Frequency Matrix
		\item Hyper-link Induced Topic Search (HITS)
	\end{itemize}
	\item Technologien:
	\begin{itemize}
		\item Statische Feature Location Technologien
		\begin{itemize}
			\item Plain Output
			\item Guided Output
		\end{itemize}
		\item Dynamische Feature Location Technologien
		\begin{itemize}
			\item Plain Output
			\item Guided Output	
		\end{itemize}
	\end{itemize}
\end{itemize}

\section{Analyseverfahren}
\subsection{Statische Analyse}
\subsection{Dynamische Analyse}
\subsection{Textuelle Analyse}

\section{Tools}
\section{Beispiel}
\section{Fazit}

\cleardoublepage

\phantomsection % benötigt für korrekte pdf-darstellung
\addcontentsline{toc}{section}{Literaturverzeichnis}
\bibliographystyle{natdin} % Din 1505 nach Lorenzen (Das konkrete Aussehen des Litverzeichnisses ist im header festgelegt)
\bibliography{literatur}  % Pfad zur *.bib-Datei (Dateiendung wird weggelassen)

\end{document}
